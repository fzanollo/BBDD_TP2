  En este trabajo práctico se presenta un ejemplo de trabajo con bases de datos noSQL (a partir de una base de datos relacional),
asi como el uso de la técnica de sharding para lograr que los datos se almacenen de forma distribuida y la técnica de map reduce,
que se utiliza para realizar consultas eficientes basadas en computo paralelo utilizada en bases de datos distribuidas.
El sistem de bases de datos a utilizar es MongoDB. MongoDB es un sistema de bases de datos orientado a documentos, guardando
la estructura de los datos en documentos tipo JSON (javascript object notation) que se almacenan en colecciones. Si estuvieramos
en el modelo relacional, cada colección tendría como una tabla y cada registro sería un documento.\\
  La estructura a utilizar para los documentos es la siguiente:
\begin{itemize}

\item 
  Facturas: En las facturas se desea almacenar, para el usuario que corresponde, nombre completo y número de id
  para poder identificarlo, la fecha para diferenciar a que mes corresponde cada factura y el detalle pertinente a cada cliente.
  Esto es, qué publicaciones realizé mediante el pago de un abono fijo (y en que monto consiste dicho abono) y las ventas que 
  realizó a comisión.
  \begin{lstlisting}[language=JavaScript]
  Facturas
  {
      usuario: {id, nombre, apellido},
      fecha: fecha
      abonoFijo: { publicaciones: [titulos], abono },
      comisiones: [ { publicacion: {titulo}, comision a pagar } ]
  }
  \end{lstlisting}
\item 
  Ventas: En este documento se extrae un resumen de los datos relacionados a una venta. Los datos que nos interesan son
  el id de la publicación, quién vendió, quién compró, que calificación se asignaron mutuamente, de cuánto fue el importe de la venta
  y que comisión corresponde cobrarle al vendedor por dicha venta.
  \begin{lstlisting}[language=JavaScript]
  Ventas:
  {
      publicacionId: id,
      importe: total de la venta,
      vendedor: {id vendedor, nombre},
      comprador: {id comprador, nombre},
      calificacionVendedor: calificacion vendedor,
      calificacionComprador: calificacion comprador,
      comision: porcentaje de comsion cobrada por la venta
  }
  \end{lstlisting}
\item 
  Publicaciones: Este documento es un resumen de cada publicación, que únicamente indica si una publicación pertenece a un servicio o a un articulo,
  el título de dicha publicación y su número de identificación.
  \begin{lstlisting}[language=JavaScript]
  Publicaciones
  {
      Id: id,
      Tìtulo: titulo,
      Tipo de publicación: servicio/articulo
  }
  \end{lstlisting}

\end{itemize}

La estructura de los documentos está definida según lo especificado en la base de datos relacional implementada en el trabajo práctico anterior,
asi como su correpondiente modelo conceptual. Los documentos contienen la información necesaria para realizar las consultas Map-Reduce
detalladas mas adelante.