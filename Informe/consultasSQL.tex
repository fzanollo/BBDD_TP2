Para generar los documentos, se debería extraer la información de la base de datos relacional generada en el trabajo práctico.
A continuación se detallan las consultas en lenguaje SQL necesarias:

  \begin{lstlisting}[language=SQL]
/*Consulta para obtener los datos para emitir la factura mensual de un usuario*/

Select		publicacionId,
		titulo,
		sum(precioPorUnidad * cantidadDeUnidades * comision) as comision
From publicaciones as p 
Join compraventa as cv on p.publicacionId= cv.publicacionId 
Join articulos as a on a.publicacionId=p.publicacionID
Join tipoPublicacion as tp on tp.idTipoPublicacion=p.tipo 
Where cv.fecha.month() >= now.month()-1 
And idTipoPublicacion= 1|2|3 
And p.email= {usuario al que corresponde la factura} 
Group by PublicaciónId, Título

#Solo para las que son oro o plata, las ruby de oriente tienen un abono fijo

  \end{lstlisting}
  \begin{description}
 \item[linea 5] Únicamente para las publicaciones oro y plata.
 \item[linea 10] Obtenemos sólo las publicaciones correspondientes al último mes.
 \item[linea 11] 1= Rubí, 2= Oro, 3= Plata
 \item[linea 12] El email (que es la clave primaria del usuario) se reemplaza por un id en los documentos.
 \item[Nota Adicional] El monto fijo a abonar por las publicaciones rubi de oriente se puede consultar aparte, ya que
 no se usa en la consulta (a diferencia de la comision)
  \end{description}
