Para generar los documentos, se debería extraer la información de la base de datos relacional generada en el trabajo práctico.
A continuación se detallan las consultas en lenguaje SQL necesarias:

  \begin{lstlisting}[language=SQL]
/*Consulta para obtener los datos para emitir la factura mensual de un usuario*/

Select		publicacionId,
		titulo,
		sum(precioPorUnidad * cantidadDeUnidades * comision) as comision
From publicaciones as p 
Join compraventa as cv on p.publicacionId= cv.publicacionId 
Join articulos as a on a.publicacionId=p.publicacionID
Join tipoPublicacion as tp on tp.idTipoPublicacion=p.tipo 
Where cv.fecha.month() >= now.month()-1 
And idTipoPublicacion= N 
And p.email= {usuario al que corresponde la factura} 
Group by PublicaciónId, Título

  \end{lstlisting}
  \begin{description}
 \item[linea 5] Únicamente para las publicaciones oro y plata.
 \item[linea 10] Obtenemos sólo las publicaciones correspondientes al último mes.
 \item[linea 11] N: 1= Rubí, 2= Oro, 3= Plata
 \item[linea 12] El email (que es la clave primaria del usuario) se reemplaza por un id en los documentos.
 \item[Nota Adicional] El monto fijo a abonar por las publicaciones rubi de oriente se puede consultar aparte, ya que
 no se usa en la consulta (a diferencia de la comision)
  \end{description}

  \begin{lstlisting}[language=SQL]
/*Consulta para obtener el detalle de las ventas*/

Select	publicacionId,
	p.Email as idVendedor,
	U.nombre as vendedor,
	Cv.email as idComprador,
	Titulo,
	Comision,
	calificacionVendedor,
	calificacionComprador,
	sum(precioPorUnidad * cantidadDeUnidades) as importe
From publicaciones as p 
Join compraventa as cv on p.publicacionId= cv.publicacionId 
Join articulos as a on a.publicacionId=p.publicacionID
Join tipoPublicacion as tp on tp.idTipoPublicacion=p.tipo
Join calificacion on cv.calificacionVendedor=calificacion.idCalifiacion
Join calificacion on cv.calificacionComprador=calificacion.idCalifiacion
And idTipoPublicacion= N
Group by 	publicacionId,
		idVendedor,
		idComprador,
		titulo,
		Comision,    
		calificacionVendedor,
		calificacionComprador

  \end{lstlisting}
  \begin{description}
 \item[linea 11] Únicamente para las publicaciones oro y plata.
 \item[linea 18] 1= Rubí, 2= Oro, 3= Plata.
  \item[Nota Adicional] El monto fijo a abonar por las publicaciones rubi de oriente se puede consultar aparte, ya que
 no se usa en la consulta (a diferencia de la comision)

  \end{description}
\newpage
   \begin{lstlisting}[language=SQL]
/*Consulta para obtener el detalle de las ventas*/

Select	publicacionId,
	p.Email as idVendedor,
	U.nombre as vendedor,
	Cv.email as idComprador,
	Titulo,
	Comision,
	calificacionVendedor,
	calificacionComprador,
	sum(precioPorUnidad * cantidadDeUnidades) as importe
From publicaciones as p 
Join compraventa as cv on p.publicacionId= cv.publicacionId 
Join articulos as a on a.publicacionId=p.publicacionID
Join tipoPublicacion as tp on tp.idTipoPublicacion=p.tipo
Join calificacion on cv.calificacionVendedor=calificacion.idCalifiacion
Join calificacion on cv.calificacionComprador=calificacion.idCalifiacion
And idTipoPublicacion= N
Group by 	publicacionId,
		idVendedor,
		idComprador,
		titulo,
		Comision,    
		calificacionVendedor,
		calificacionComprador

  \end{lstlisting}
  \begin{description}
 \item[lineas 3 y 5] El email (que es la clave primaria del usuario) se reemplaza por un id en los documentos.
 \item[linea 11] Únicamente para las publicaciones oro y plata.
 \item[linea 18] 1= Rubí, 2= Oro, 3= Plata.
 \item[Nota adicional] Por claridad, se omiten de la consulta el nombre y el apellido del comprador/vendedor.
 Estos pueden obtenerse trivialmente consultando la tabla de usuarios usando la clave primaria de los usuarios
 (el email), que sí está de manera explícita en la consulta.

  \end{description}
   \begin{lstlisting}[language=SQL]
/*Consulta para obtener el resumen de las publicaciones*/

Select	publicacionId,
	Titulo,
	TipoPublicacion
From publicaciones as p 

  \end{lstlisting}
